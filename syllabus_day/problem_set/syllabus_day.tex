\documentclass[11pt]{article}
\usepackage{amsmath, amssymb, graphicx, hyperref}
\usepackage{enumitem}
\setlist{nosep}
\usepackage[margin=1in]{geometry}

\title{ Regular Expressions in R (Functions \& Validation)}
\author{ }
\date{ }



\begin{document}
\maketitle

\section*{Conceptual Questions}
Please write three to ten sentence explanations for each of the following questions. \textbf{You are only required to answer ONE of the two questions below.} \bigskip
 
\begin{enumerate}

\item In a data science workflow, why is it useful to wrap data-cleaning logic into functions (rather than writing one-off code blocks)? In your answer, connect your explanation to regex-based parsing/validation and mention one way you would test that your function behaves correctly.

\item When writing regex for data extraction, what is the difference between (i) identifying a match, (ii) capturing components of a match, and (iii) replacing a match? Use a concrete example from either phone-number standardization or password validation to illustrate your answer.

  \end{enumerate}


\section*{Applied Exercises}
Use the code in the week's code tutorial and the lecture slides to answer the following questions.\bigskip

  \begin{enumerate}
  \setcounter{enumi}{2}

\item \textbf{Phone Number Formatting:}
Create a regex pattern to identify and format phone numbers in text. Your function should work for different formats:
\begin{itemize}
  \item \texttt{(123) 456-7890}
  \item \texttt{123-456-7890}
  \item \texttt{1234567890}
\end{itemize}
and standardize them to: \texttt{(XXX) XXX-XXXX}.

\bigskip
\noindent \textbf{Requirements:}
\begin{itemize}
  \item Implement \texttt{format\_phone\_numbers(text)} using \texttt{stringr} (e.g., \texttt{str\_replace\_all()}).
  \item Demonstrate that it works on a character vector with \textbf{at least 6} test strings, including at least one string that contains \textbf{two} phone numbers.
  \item Show the \textbf{before} and \textbf{after} output clearly (print a small table is fine).
\end{itemize}

\bigskip
\noindent \textbf{Starter skeleton (you may copy/paste):}
\begin{verbatim}
# TODO: Implement this function.
format_phone_numbers <- function(text) {
  # pattern <- ...
  # Use str_replace_all / str_replace to standardize
}
\end{verbatim}

\item \textbf{Password Strength Checker:}
Implement a function that uses regex to check the strength of a password.

\bigskip
\noindent \textbf{Criteria:}
\begin{itemize}
  \item At least 8 characters long
  \item At least one uppercase letter
  \item At least one lowercase letter
  \item At least one digit
  \item At least one special character (\texttt{@\$!\%*?\&})
\end{itemize}

\bigskip
\noindent \textbf{Requirements:}
\begin{itemize}
  \item Implement \texttt{check\_password\_strength(password)} using \texttt{stringr} (e.g., multiple \texttt{str\_detect()} checks, or a single combined pattern).
  \item Create a test vector with \textbf{at least 8} passwords (mix of valid/invalid).
  \item Output a small results table with columns like \texttt{password} and \texttt{is\_strong} (and optionally a column indicating what criterion failed).
\end{itemize}

\bigskip
\noindent \textbf{Starter skeleton (you may copy/paste):}
\begin{verbatim}
# TODO: Implement this function.
check_password_strength <- function(password) {
  # Use str_detect with multiple patterns (or a single combined pattern)
}
\end{verbatim}

\item \textbf{Challenge Question (Optional --- if you finish early):}
Write a minimal set of unit tests for your two functions using \texttt{testthat}.
\begin{itemize}
  \item Create at least \textbf{5} tests for \texttt{format\_phone\_numbers()} and at least \textbf{5} tests for \texttt{check\_password\_strength()}.
  \item Include at least one edge case for each (e.g., numbers embedded in longer digit strings; passwords that fail exactly one rule).
  \item Show that your tests run without failing.
\end{itemize}

\end{enumerate}

\end{document}
